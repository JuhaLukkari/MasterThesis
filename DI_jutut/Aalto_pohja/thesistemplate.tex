%%%%%%%%%%%%%%%%%%%%%%%%%%%%%%%%%%%%%%%%%%%%%%%%%%%%%%%%%%%%%%%%%%%%
%%%%%%%%%%%%%%%%%%%%%%%%%%%%%%%%%%%%%%%%%%%%%%%%%%%%%%%%%%%%%%%%%%%%
%%                                                                %%
%% An example for writting your thesis using LaTeX                %%
%% Original version by Luis Costa,  changes by Perttu Puska       %%
%% Support for Swedish added 15092014                             %%
%%                                                                %%
%% This example consists of the files                             %%
%%         thesistemplate.tex (versio 2.01)                       %%
%%         opinnaytepohja.tex (versio 2.01) (for text in Finnish) %%
%%         aaltothesis.cls (versio 2.01)                          %%
%%         kuva1.eps                                              %%
%%         kuva2.eps                                              %%
%%         kuva1.pdf                                              %%
%%         kuva2.pdf                                              %%
%%                                                                %%
%%                                                                %%
%% Typeset either with                                            %%
%% latex:                                                         %%
%%             $ latex opinnaytepohja                             %%
%%             $ latex opinnaytepohja                             %%
%%                                                                %%
%%   Result is the file opinnayte.dvi, which                      %%
%%   is converted to ps format as follows:                        %%
%%                                                                %%
%%             $ dvips opinnaytepohja -o                          %%
%%                                                                %%
%%   and then to pdf as follows:                                  %%
%%                                                                %%
%%             $ ps2pdf opinnaytepohja.ps                         %%
%%                                                                %%
%% Or                                                             %%
%% pdflatex:                                                      %%
%%             $ pdflatex opinnaytepohja                          %%
%%             $ pdflatex opinnaytepohja                          %%
%%                                                                %%
%%   Result is the file opinnaytepohja.pdf                        %%
%%                                                                %%
%% Explanatory comments in this example begin with                %%
%% the characters %%, and changes that the user can make          %%
%% with the character %                                           %%
%%                                                                %%
%%%%%%%%%%%%%%%%%%%%%%%%%%%%%%%%%%%%%%%%%%%%%%%%%%%%%%%%%%%%%%%%%%%%
%%%%%%%%%%%%%%%%%%%%%%%%%%%%%%%%%%%%%%%%%%%%%%%%%%%%%%%%%%%%%%%%%%%%

%% Uncomment one of these:
%% the 1st when using pdflatex, which directly typesets your document in
%% pdf (use jpg or pdf figures), or
%% the 2nd when producing a ps file (use eps figures, don't use ps figures!).
\documentclass[english,12pt,a4paper,pdftex,elec,utf8]{aaltothesis}
%\documentclass[english,12pt,a4paper,dvips]{aaltothesis}

%% To the \documentclass above
%% specify your school: arts, biz, chem, elec, eng, sci
%% specify the character encoding scheme used by your editor: utf8, latin1

%% Use one of these if you write in Finnish (see the Finnish template):
%%
%\documentclass[finnish,12pt,a4paper,pdftex,elec,utf8]{aaltothesis}
%\documentclass[finnish,12pt,a4paper,dvips]{aaltothesis}

\usepackage{graphicx}
\usepackage{listings}

%% Use this if you write hard core mathematics, these are usually needed
\usepackage{amsfonts,amssymb,amsbsy}

%% Use the macros in this package to change how the hyperref package below 
%% typesets its hypertext -- hyperlink colour, font, etc. See the package
%% documentation. It also defines the \url macro, so use the package when 
%% not using the hyperref package.
%%
%\usepackage{url}

%% Use this if you want to get links and nice output. Works well with pdflatex.
\usepackage{hyperref}
\hypersetup{pdfpagemode=UseNone, pdfstartview=FitH,
  colorlinks=true,urlcolor=red,linkcolor=blue,citecolor=black,
  pdftitle={Default Title, Modify},pdfauthor={Your Name},
  pdfkeywords={Modify keywords}}


%% All that is printed on paper starts here
\begin{document}

%% Change the school field to specify your school if the automatically 
%% set name is wrong
% \university{aalto-yliopisto}
\university{aalto University}
% \school{Sähkötekniikan korkeakoulu}
\school{School of Electrical Engineering}

%% Only for B.Sc. thesis: Choose your degree programme. 
%\degreeprogram{Electronics and electrical engineering}
%%

%% ONLY FOR M.Sc. AND LICENTIATE THESIS: Specify your department,
%% professorship and professorship code. 
%%
\department{Department of Radio Science and Technology}
\professorship{Space Science and Technology}
%%

%% Valitse yksi näistä kolmesta
%%
%% Choose one of these:
%\univdegree{BSc}
\univdegree{MSc}
%\univdegree{Lic}

%% Your own name (should be self explanatory...)
\author{Juha-Matti Lukkari}

%% Your thesis title comes here and again before a possible abstract in
%% Finnish or Swedish . If the title is very long and latex does an
%% unsatisfactory job of breaking the lines, you will have to force a
%% linebreak with the \\ control character. 
%% Do not hyphenate titles.
%% 
\thesistitle{Automated functional testing and control of Suomi 100 satellite systems and payload instrument control software}

\place{Espoo}

%% For B.Sc. thesis use the date when you present your thesis. 
%% 
%% Kandidaatintyön päivämäärä on sen esityspäivämäärä! 
\date{16.1.2015}

%% B.Sc. or M.Sc. thesis supervisor 
%% Note the "\" after the comma. This forces the following space to be 
%% a normal interword space, not the space that starts a new sentence. 
%% This is done because the fullstop isn't the end of the sentence that
%% should be followed by a slightly longer space but is to be followed
%% by a regular space.
%%
\supervisor{Prof.\ Esa Kallio} %{Prof.\ Pirjo Professori}

%% B.Sc. or M.Sc. thesis advisors(s). You can give upto two advisors in
%% this template. Check with your supervisor how many official advisors
%% you can have.
%%
%\advisor{Prof.\ Pirjo Professori}
\advisor{D.Sc.\ (Tech.) Antti Kestilä}
\advisor{M.Sc.\ Juha Itkonen}

%% Aalto logo: syntax:
%% \uselogo{aaltoRed|aaltoBlue|aaltoYellow|aaltoGray|aaltoGrayScale}{?|!|''}
%%
%% Logo language is set to be the same as the document language.
%% Logon kieli on sama kuin dokumentin kieli
%%
\uselogo{aaltoRed}{''}

%% Create the coverpage
%%
\makecoverpage


%% Note that when writting your master's thesis in English, place
%% the English abstract first followed by the possible Finnish abstract

%% English abstract.
%% All the information required in the abstract (your name, thesis title, etc.)
%% is used as specified above.
%% Specify keywords
%%
%% Kaikki tiivistelmässä tarvittava tieto (nimesi, työnnimi, jne.) käytetään
%% niin kuin se on yllä määritelty.
%% Avainsanat
%%
\keywords{For keywords choose concepts that are central to your thesis}
%% Abstract text
\begin{abstractpage}[english]
 Large portion of launched Cubesats have failed early on their missions. Complete lack or inadequate system level functional testing of the satellites has been thought of being one large contributor to these failures. Software failures could be another major contributor. This thesis represents potential solutions to mitigating these issues by the use of free open-source automated test frameworks.

  Your abstract in English. Try to keep the abstract short; approximately 
  100 words should be enough. The abstract explains your research topic, 
  the methods you have used, and the results you obtained.  
  Your abstract in English. Try to keep the abstract short; approximately 
  100 words should be enough. The abstract explains your research topic, 
  the methods you have used, and the results you obtained.  
\end{abstractpage}

%% Force a new page so that the possible English abstract starts on a new page
%%
%% Pakotetaan uusi sivu varmuuden vuoksi, jotta 
%% mahdollinen suomenkielinen ja englanninkielinen tiivistelmä
%% eivät tule vahingossakaan samalle sivulle
\newpage
%
%% Abstract in Finnish.  Delete if you don't need it. 
\thesistitle{Suomi100 satelliitin järjestelmien ja hyötykuorman instrumentin ohjausohjelmiston automaattinen funktionaalinen testaus ja ohjaus}
\advisor{TkT Olli Ohjaaja}
\degreeprogram{Electronics and electrical engineering}
\department{Radiotieteen ja -tekniikan laitos}
\professorship{Avaruus tiede ja tekniikka}
%% Avainsanat
\keywords{Vastus, Resistanssi,\\ Lämpötila}
%% Tiivistelmän tekstiosa
\begin{abstractpage}[finnish]
  Tiivistelmässä on lyhyt selvitys (noin 100 sanaa)
  kirjoituksen tärkeimmästä sisällöstä: mitä ja miten on tutkittu,
  sekä mitä tuloksia on saatu. 
  Tiivistelmässä on lyhyt selvitys (noin 100 sanaa)
  kirjoituksen tärkeimmästä sisällöstä: mitä ja miten on tutkittu,
  sekä mitä tuloksia on saatu. 

  Tiivistelmässä on lyhyt selvitys (noin 100 sanaa)
  kirjoituksen tärkeimmästä sisällöstä: mitä ja miten on tutkittu,
  sekä mitä tuloksia on saatu. 
  Tiivistelmässä on lyhyt selvitys (noin 100 sanaa)
  kirjoituksen tärkeimmästä sisällöstä: mitä ja miten on tutkittu,
  sekä mitä tuloksia on saatu. 
  Tiivistelmässä on lyhyt selvitys (noin 100 sanaa)
  kirjoituksen tärkeimmästä sisällöstä: mitä ja miten on tutkittu,
  sekä mitä tuloksia on saatu. 
\end{abstractpage}

%% Force new page so that the Swedish abstract starts from a new page
\newpage
%
%% Swedish abstract. Delete if you don't need it. 
%% 
\thesistitle{Arbetets titel}
\advisor{TkD Olli Ohjaaja} %
\degreeprogram{Electronik och electroteknik}
\department{Institutionen för radiovetenskap och -teknik}%
\professorship{Kretsteori}  %
%% Abstract keywords
\keywords{Nyckelord p\aa{} svenska,\\ Temperatur}
%% Abstract text
\begin{abstractpage}[swedish]
 Sammandrag p\aa{} svenska.
 Try to keep the abstract short, approximately 
 100 words should be enough. Abstract explains your research topic, 
 the methods you have used, and the results you obtained.  
\end{abstractpage}

%% Preface
%%
%% Esipuhe 
\mysection{Preface}
%\mysection{Esipuhe}
I want to thank Professor Esa Kallio
and my instructors Antti Kestilä and Juha Itkonen for their 
good guidance.\\

\vspace{5cm}
Otaniemi, 16.1.2015

\vspace{5mm}
{\hfill Eddie E.\ A.\ Engineer \hspace{1cm}}

%% Force new page after preface
%%
%% Pakotetaan varmuuden vuoksi esipuheen jälkeinen osa
%% alkamaan uudelta sivulta
\newpage


%% Table of contents. 
\thesistableofcontents


%% Symbols and abbreviations
\mysection{Symbols and abbreviations}

\subsection*{Symbols}

\begin{tabular}{ll}
$\mathbf{B}$  & magnetic flux density  \\
$c$              & speed of light in vacuum $\approx 3\times10^8$ [m/s]\\
$\omega_{\mathrm{D}}$    & Debye frequency \\
$\omega_{\mathrm{latt}}$ & average phonon frequency of lattice \\
$\uparrow$       & electron spin direction up\\
$\downarrow$     & electron spin direction down
\end{tabular}

\subsection*{Operators}

\begin{tabular}{ll}
$\nabla \times \mathbf{A}$              & curl of vectorin $\mathbf{A}$\\
$\displaystyle\frac{\mbox{d}}{\mbox{d} t}$ & derivative with respect to 
variable $t$\\[3mm]
$\displaystyle\frac{\partial}{\partial t}$  & partial derivative with respect 
to variable $t$ \\[3mm]
$\sum_i $                       & sum over index $i$\\
$\mathbf{A} \cdot \mathbf{B}$    & dot product of vectors $\mathbf{A}$ and 
$\mathbf{B}$
\end{tabular}

\subsection*{Abbreviations}

\begin{tabular}{ll}
OBC         & On board computer \\
EPS      & Electric power system \\
BCS        & Bardeen-Cooper-Schrieffer \\ %% dash between the names
DC         & direct current \\
TEM        & transverse eletromagnetic
\end{tabular}


%% Tweaks the page numbering to meet the requirement of the thesis format:
%% Begin the pagenumbering in Arabian numerals (and leave the first page
%% of the text body empty, see \thispagestyle{empty} below).
%% Additionally, force the actual text to begin on a new page with the 
%% \clearpage command.
%% \clearpage is similar to \newpage, but it also flushes the floats (figures
%% and tables).
%% There is no need to change these
%%
\cleardoublepage
\storeinipagenumber
\pagenumbering{arabic}
\setcounter{page}{1}


%% Text body begins. Note that since the text body
%% is mostly in Finnish the majority of comments are
%% also in Finnish after this point. There is no point in explaining
%% Finnish-language specific thesis conventions in English. Someday 
%% this text will possibly be translated to English.
%%
\section{Introduction}
%\section{Introduction}

%% Ensimm\"ainen sivu tyhj\"aksi
%% 
%% Leave first page empty
\thispagestyle{empty}

Cubesats have in recent years emerged as a new viable platform for carrying out space missions. They usually are produced by Universities and over 200 have already been launched. Yet, many of those missions have ended in failure due to various reasons.\par 
The satellite involved in our research is called \textit{Suomi100}.

%% Esimerkki luettelosta. Lyhyt ajatusviiva on k\"ayt\"oss\"a
%% luettelossa, ja se on pituudeltaan
%% en dash. Merkit\"a\"an latex-koodissa --. 


\begin{itemize}
\item[--]About cubesats in general
\item[--]Failures with cubesats
\item[--]Suomi100 satellite mission
\item[--]Doing test automation for Suomi100 satellite
\item[--]Robot framework, CUnit
\end{itemize}


%% Opinn\"aytteess\"a jokainen osa alkaa uudelta sivulta, joten \clearpage
%%
%% In a thesis, every section starts a new page, hence \clearpage
\clearpage

\section{Background}
%\section{Aikaisempi tutkimus}

\begin{itemize}
\item[--]Generally about spacecraft failures
\item[--]Failures with Cubesats
\item[--]Cubesat failure database, Swartwout research and conclusions -> Inadequate functional system level testing
\item[--]Earlier research done on flight software reliability
\item[--]Earlier research done on system level functional testing
\item[--]Testing done on satellites with big budgets vs testing on Cubesats
\item[--]About different testing methodologies, black box, white box
\end{itemize}

"Looking at the failure reports more closely, a com-
mon thread is discovered, accounting for almost half
of all failures: a configuration or interface failure be-
tween communications hardware (27\%), the power
subsystem (14\%) and the flight processor (6\%). Typical
examples of such failures: batteries and/or solar panels
not connected properly to the power bus; insufficient
power generation to operate the transmitter at a level
needed to close the link; and unrecoverable processor
errors. These can be classified as failures in functional
integration; the spacecraft was not operated in a flight-
equivalent state before launch, and thus these easily-
caught mistakes were not discovered. Though this
allegation obviously cannot be proven, it is strongly be-
lieved that a large fraction of the “no contact” failures
is due to poor functional integration"\cite{Swart1}\par
"As  noted  above,  it  is  believed  that  the  failure-rate  
problem  is  solvable,  given  that  most  failures  can  be  
traced back to insufficient system-level functional testing  on  the  ground.  Furthermore,  it  is  thought  that  a  
“day  in  the  life”  operational  demonstration  is  just  as  
essential  as  vibration  testing  to  certify  a  CubeSat  for  
flight.  Operational  tests  that  demonstrate  startup  sequences,  power  management  and  graceful  recovery  
from resets are all necessary"\cite{Swart1}\\

\begin{itemize}
\item[--]Just as vibrational testing is done by automated machines to verify that the satellite can withstand the launch, automated software and system tests can verify that the satellite operates as expected. Robot framework is free and open-source tool for automated testing, so it is suitable for Cubesat projects.
\item[--]Present in Results section some philosophy or methodology for future Cubesat projects on how to do functional system level testing in practice? Some API on the satellite (like csp-client) -> code to send commands to the API automatically -> Some way or some script to run the commands -> Some way to gather information from the satellite on what is going on (csp-client prints to the console).
\end{itemize}



%% Osan hienojaottelua alaosiin, eik\"a v\"altt\"am\"att\"a edes tarpeen,
%% t\"ass\"a vain esimerkkin\"a. K\"ayt\"a harkintasi mukaan
%% osan jaottelua, joskus alaotsikot selvent\"av\"at asioita ja
%% joskus vain sirpaloittavat tarpeettomasti teksti\"a.
%%  Jaottelu menee seuraavasti:
%% \section{osan otsikko} 
%% \subsection{alaotsikko}
%% \subsubsection{ala-alaotsikko}
%% T\"at\"a pitem\"alle ei pid\"a jaotella. 
%%
%% Three levels of hierarchy in sectioning should be enough



\clearpage

\section{Automated testing done on Suomi100 satellite}
%\section{Tutkimusaineisto ja -menetelm\"at}
\begin{itemize}
\item[--]About Suomi100 subsystems
\item[--]About Gomspace software and instrument software
\item[--]About functional testing
\item[--]About unit testing
\item[--]Robot framework, CUnit
\item[--]Operation modes of the satellite
\item[--]Testing environment, gomspace API client
\item[--]Automating control of Suomi100 with Robot and gomspace client
\end{itemize}
 

\clearpage

\section{Results}
%\section{Tulokset}
\begin{itemize}
\item[--]Test results
\item[--]How operation modes behaved during tests
\item[--]How the different subsystems behaved during tests
\item[--]How the instrument behaved during tests
\item[--]Importantly: What issues with the software and subsystems were found with the tests and what corrections were implemented based on the test results
\item[--]AFTER LAUNCH:
\item[--]Operation modes
\item[--]Things that worked, commands that went through
\item[--]About failures if those happened, software crashes
\item[--]What was found during testing and comparison to how the satellite performed in orbit. Did the implemented changes to the software improve reliability.
\end{itemize}


%% Huomaa seuraavassa kappaleessa lainausmerkkien ulkopuolella piste, 
%% koska piste ei lopeta lainattua tekstinp\"atk\"a\"a.
%% Jos lainattu tekstinp\"atk\"a loppuu v\"alimerkkiin, tulee v\"alimerkki
%% lainausmerkkien sis\"alle: 
%% "Et tu, Brute?" sanoi Caesar kuollessaan.


\clearpage

\section{Summary} 
%\section{Yhteenveto}

\begin{itemize}
\item[--]Suomi100 mission
\item[--]Failures with cubesats
\item[--]We tried to improve overall system reliability with automated functional tests
\item[--]We tried to improve instrument software reliability with automated unit and functional tests
\item[--]Did we improve the functionality and reliability of the satellite?

\end{itemize}



\clearpage
%% L\"ahdeluettelo
%%
%% \phantomsection varmistaa, ett\"a hyperref-paketti latoo hypertekstilinkit
%% oikein.
%%
%% The \phantomsection command is nessesary for hyperref to jump to the 
%% correct page, in other words it puts a hyper marker on the page.

\phantomsection
\addcontentsline{toc}{section}{\refname}
%\addcontentsline{toc}{section}{References}
\begin{thebibliography}{99}

%% Alla pilkun j\"alkeen on pakotettu oikea v\"ali \<v\"alily\"onti>-merkeill\"a.
\bibitem{Swart1} Swartwout, Michael\\
  \textit{The First One Hundred CubeSats: A Statistical Look}  Journal of small satellites, 2013.



\end{thebibliography}

%% Appendices
%% Liitteet
\clearpage

\thesisappendix

\section{Esimerkki liitteest\"a\label{LiiteA}}
\begin{lstlisting}[language=C]
/*
Radio Operation modes
*/
#include <radio.h>

#include <inttypes.h>
#include <string.h>
#include <stdio.h>
#include <stdint.h>
#include <string.h>
#include <ctype.h>
#include <stdlib.h>
#include <pthread.h>
#ifdef linux
#include <time.h>
#else
#include <FreeRTOS.h>
#include <task.h>
#endif

#include <util/console.h>
#include <util/log.h>
#include <util/color_printf.h>
#include <util/clock.h>

#include <csp/csp.h>
#include <csp/csp_endian.h>
#include <csp/arch/csp_thread.h>
#include "radio_config.h"
#include "libradio.h"
#include "radio_calculation.h"
#include "radio_property.h"


// Syntax for different arguments val1;val2;val3; 
// NOTE: Frequency in kilohertz
void raw_data_mode(char *mode_args)
{
	uint32_t t_start = 0;
	uint16_t f_const = 0; 
	uint32_t n_times = 0;
	uint32_t t_sleep = 0;
	uint8_t ferrite = 0;
	char temp[32] = {0};
	unsigned int i = 0;
	unsigned int j = 0;
	unsigned int argc = 0;
	while(i < strlen(mode_args))
	{
		if(mode_args[i] != ';')
		{
			temp[j] = mode_args[i];
			j++;
		}
		else
		{
			if(argc == 0)
				t_start = strtol(temp, NULL, 10);
			else if(argc == 1)
				f_const = strtol(temp, NULL, 10);
			else if(argc == 2)
				t_sleep = strtol(temp, NULL, 10);
			else if(argc == 3)
				n_times = strtol(temp, NULL, 10);
			else if(argc == 4)
				ferrite = strtol(temp, NULL, 10);
			j = 0;
			argc++;
			printf("temp:%s\n", temp);
			memset(temp, 0 ,sizeof(temp));
			if(argc > 4)
				break;
		}
		printf("%c", mode_args[i]);
		i++;
	}
	printf("Arguments received for operation:%d %d %d %d %d\n", t_start, f_const, t_sleep, n_times, ferrite);
	// If values invalid for some reason, use these default values
	if(t_start < 0)
		t_start = 0;
	if(f_const < 149 || f_const > 50000)
		f_const = 5000;
	if(t_sleep < 1)
		t_sleep = 100;
	if(n_times < 1)
		n_times = 100000;
	if(ferrite < 0 || ferrite > 1)
		ferrite = 0;
	printf("Arguments corrected for operation:%d %d %d %d %d\n", t_start, f_const, t_sleep, n_times, ferrite);
	// sleep or vTaskDelay
	#ifdef linux
		usleep(t_start*10);
	#else
		vTaskDelay(t_start);
	#endif
	printf("Radio raw data mode commencing\n");
	//Choose ferrite with the switch command, how?
	
	// Open file for appending/writing
	// in FreeRTOS create file in /flash/...

	// Read from spi
	#ifdef linux
		uint32_t start_time = 0;
		uint32_t end_time = 0;
	#else
		timestamp_t *start_time = malloc(sizeof(timestamp_t));
		timestamp_t *end_time = malloc(sizeof(timestamp_t));
	#endif

	#ifdef linux
		start_time = (int)time(NULL);
	#else
		clock_get_time(start_time);
	#endif
	uint16_t *data = calloc(10, sizeof(int));
	FILE *dfp;
	// Get system time to the filename, FreeRTOS has some command ?
	char filename[64] = {0};
	char filetime[64] = {0};
	#ifdef linux
		sprintf(filetime, "%d", (int)time(NULL));
	#else
		sprintf(filetime, "%d", start_time->tv_sec);
	#endif	
	strcat(filename, "m1_");
	strcat(filename, filetime);
	strcat(filename, ".csv");
	if((dfp = fopen(filename, "a")) == NULL)
		fprintf(stderr, "Could not open %s\n", filename);
	fprintf(dfp, "Val,Freq\n");

	// si chip takes frequency in khz
	radio_am_tune_freq("0", f_const, 0);
	// Wait for chip stabilization
	#ifdef linux
		usleep(t_sleep*10);	// usleep is deprecated in POSIX, nanotime is preferred
	#else
		vTaskDelay(t_sleep);	// Milliseconds
	#endif
	for(unsigned int n = 0; n < n_times; n++)
	{
		//spi_16read(data);
		 
		// Append to file
		fprintf(dfp, "%d,%d\n", n, f_const);	
	}
	#ifdef linux
		end_time = (int)time(NULL);
		fprintf(dfp, "\nStart: %d End: %d\n", start_time, end_time);	
	#else
		clock_get_time(end_time);
		fprintf(dfp, "\nStart: %d End: %d\n", start_time->tv_sec, end_time->tv_sec);	
		free(start_time);
		free(end_time);
	#endif	

	fclose(dfp);	
	free(data);
} 
\end{lstlisting}



\clearpage
\section{Toinen esimerkki liitteest\"a\label{LiiteB}}
\begin{lstlisting}
*** Settings ***
Library		libclient.py
Suite Teardown	Client Close	${sock}		{proc}

*** Test Cases ***

Test Setup
	${proc}=		Client Start	/home/juha/S100/EGSE/EGSE/csp-client-v1.1/build/csp-client	-a 8 -d /dev/ttyUSB0 -b 500000
	#${proc}=		Client Start	/home/juha/S100/EGSE/EGSE/csp-client-v1.1/build/csp-client	-a 10 -z localhost
	Set Suite Variable	${proc}
	Sleep			5
	${sock}=		Connect Socket	s100-juha	5000
	Set Suite Variable	${sock}


Target Mode
	[Documentation]		The payload radio performs several sweeps over the entire frequency range.
	[Tags]			OPMODE-TARGET
	Store Client Responses	${proc}	 Target Mode
	# Are we sure that we start the thread in the satellite?
	#Run Radio Mode		${sock}  ${proc}  /home/juha/S100/confs/radio_params.cfg  /home/juha/S100/confs/radio_props.cfg  3  0;0;0;0;0;0;0;0;0;
	Run Radio Mode		${sock}  ${proc}  /flash/radio_params.cfg  /flash/radio_props.cfg  3  0;0;0;0;0;0;0;0;0;
	Sleep			2
	# Add proper hk and beacon commands here with flight planner
	# csp-client doesnt have flight planner
	#Get HK			${sock}		${proc}		30
	#Send Beacon		${sock}		${proc}		10	
	Verify Radio Results	Target Mode
	Send Message		${sock}		${proc}		exit_client
	Close Connection	${sock}

\end{lstlisting}


\end{document}
